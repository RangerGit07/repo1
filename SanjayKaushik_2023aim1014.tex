\documentclass[a0,portrait ] {article}
\usepackage [top=2.5cm, bottom=2.5cm, left=3cm, right=3cm]{geometry}
\usepackage{multicol}




\begin{document}
\begin{center}

%\begin{minipage}[t]{1.3\linewidth}
\centering \large \textbf{Joint Optimization of Trajectory Planning and Task Scheduling in \\ Heterogeneous Multi-UAV System \\}
\vspace{1cm}
\centering \textbf{ Sanjay Kaushik 2023aim1014\\}
%\end{minipage}
\end{center}
\\
\begin{multicols}{2}
\vspace{1cm}
\section*{Abstract} \label{sec:1}
%\vspace{1cm}
The use of unmanned aerial vehicles (UAV) as a new sensing paradigm is emerging for surveillance and tracking applications, especially in the infrastructure-less environment. One such application of UAVs is in the construction industry where currently prevalent manual progress tracking results in schedule delays and cost overruns. In this paper, we develop a heterogeneous multi-UAV frame work for progress tracking of large construction sites. The proposed framework consists of \emph{Edge\_UAV} which coordinates the data relay of the visual sensor-equipped Inspection UAVs (I UAVs) to the cloud. Our framework jointly takes into consideration the trajectory optimization of the Edge UAV and the stability of system queues. In particular, we develop a Distance and Access Latency Aware Trajectory (DLAT) optimization that generates a fair ac cess schedule for I UAV s. In addition, a Lyapunov based online optimization ensures the system stability of the average queue backlogs for data offloading tasks. Through a message based mechanism, the coordination between the set of I UAVs and Edge UAV is ensured without any dependence on any central entity or message broadcasts. The performance of our proposed framework with joint optimization algorithm is validated by extensive simulation results in different parameter settings.\\
\hspace{1cm} \textbf{Keyword}: Path Planning, Task Scheduling, Data Of floading, Construction Site Monitoring, Unmanned Aerial Vehicles (UAVs), Lyapunov Optimization


\section{Introduction} \label{sec:2}
The unmanned aerial vehicle (UAV) based solutions are
 emerging in various domains such as wireless sensing [1],
 payload delivery [2], precision agriculture [3], help and
 rescue operations [4], etc. Moreover, with the current
 trend of automation, sensing and information exchange
 in Industry 4.0, UAV based applications are also finding
 their place in the construction industry especially for re
source tracking and progress monitoring using aerial im
agery. Such solutions are helpful in infrastructure-less
 large construction sites as they provide ease of deploy
ment, quick access to the ground-truth data and higher
 reachability and coverage [5]. Further, the autonomous
 or semi-autonomous UAV based solutions could facilitate
 progress monitoring, building inspections (for cracks or
 other defects), safety inspections (to find any environ
mental hazards) and many more construction-specific au
dits automatically. The UAV based visual monitoring of
 under-construction projects also allows simultaneous ob
servability of ground-truth data by different collaborating
 entities. Availability of such data and information helps
 in timely assessments that could reduce schedule delays,
 cost overruns, resource wastage and financial losses which
 are not uncommon in construction projects.
 A plausible solution to address the aforementioned
 challenges could be a Mobile Edge Computing (MEC)
 [6] based heterogeneous multi-UAV framework. Such a
 framework along with the prior geometric knowledge avail
able about the construction site as gathered from a Build
ing Information Model (BIM)[7] could help create an ef
fective multi-UAV based visual monitoring system for con
struction sites. As for any constrained environment, the
 optimization of computational resources is central to de
velop a solution. The integration of UAVs and MEC into
 a single framework could facilitate that with efficient data
 collection/processing from the UAV based dynamic sen
sors in infrastructure-less environments [8]. In addition,
 an MEC based framework can help to perform partial
 computation offloading wherein a part of data is processed
 by the UAVs while the rest gets offloaded to the cloud.
 An MECbased UAV framework is not new and the de
ployment of the UAVs as base stations or edge servers
 is widely studied [9, 10]. These studies reflect on the f
 lexibility in deployment of UAV based edge computing
 components. However, there is a problem of buffer over
f
 low of UAVs due to the limited on-board processing and
 the shared bandwidth to transfer data to the cloud which
 leads to instability in the system. In addition, the dy
namic nature of such systems with varying data traffic
 and continuous movement of UAVs makes it difficult to
 stabilize or control the system in a deterministic man
ner. Researchers have used online Lyapunov optimization
 [11] to address such system instabilities. Lyapunov opti
mization considers the stability of the system with time
 varying data and optimizes time averages of system utility
 and queue backlogs.
 In this paper, we address the challenges of deploying
 a heterogeneous multi-UAV system for construction site
 monitoring by the joint optimization of UAV trajectory
 planning and data offloading task scheduling. The pro
posed framework employs two types of UAVs viz. Inspec
tion UAVs I UAVsandMobileEdgeUAVs\emph{(Edge\_UAV)}.
 While the former is deployed as visual sensors to collect
 visual data from different locations of the site, the latter
 interacts and collects data from I UAV s, and offloads the
 same to the cloud. The core objective of the framework
 is to minimize the total energy consumption of the sys
tem while considering the data queue backlogs of I UAV s
 and Edge UAV and also jointly optimizing the trajectory
 of the Edge UAV in accordance with the trajectories of
 I UAVs having minimum access latency and travel dis
tance. The online resource management such as transmis
sion power and processor frequency of the Edge UAV is
 evolved using Lyapunov optimization (as in [12]).
 The rest of the paper is organised as follows: Section
 2 presents the proposed heterogeneous multi-UAV frame
work for construction site monitoring. The overall system
 objective is discussed in Sections 3. Sections 4 and 5 dis
cuss the trajectory optimization and Lyapunov based sys
tem stability, respectively. The simulation setup has been
 presented in Section 6. Section 7 discusses the results
 gathered from the experiments while Section 8 concludes
 the paper.

 \section{Heterogeneous Multi-UAV Frame work} \label{sec:3}
 Figure ?? depicts the overall multi-UAV framework with
 all its components. The system consists of two hetero
geneous UAVs i.e. a set of Inspection UAVs I UAV =
 {I UAV1,I UAV2,I UAV3,.....,I UAVN} and a Mobile
 Edge UAV (Edge UAV). I UAVs are smaller in size and
 are more agile. They collect visual data from a set of
 Point of Interests (PoIs) denoted as L = {l1,l2,l3....lk}
 across the construction site. As the construction sites are
 infrastructure-less environments, there are limited Access
 Points (AP) available for connectivity to the cloud. Fur
ther, the I UAVs possess limited connectivity range that
 makes it difficult for them to transfer data to cloud di
rectly. In addition, the I UAV s move in the 3D Cartesian
 coordinate system. The Edge UAV, which is larger in size
 and possesses higher computational capabilities, coordi
nates with the I UAVs to relay the data (after partially
 processing the same) to the cloud. Edge UAV always
 maintains a constant height and thus its trajectory lies in
 an horizontal plane.


\cite{mozaffari2019tutorial}
\cite{ruggiero2018aerial}
\cite{boursianis2020internet}
\cite{waharte2010supporting}
\cite{hamledari2018uav}
\cite{mao2017survey}
\cite{golparvar2011integrated}
\cite{nguyen2020towards}
\cite{wan2019towards}
\cite{abbas2017mobile}
\cite{neely2010stochastic}
\cite{zhang2018stochastic}
\cite{al2014optimal}
\cite{lu2018beyond}
\cite{MichelNeeley2018}
\cite{}
\bibliographystyle{acm}
\bibliography{bibs}

\end{multicols}

 \end{document}